% \documentclass[manuscript]{geophysics}
\documentclass[paper,twocolumn,twoside]{geophysics}
% \documentclass[paper]{geophysics}
% \documentclass[paper, revised]{geophysics}
% \documentclass[manuscript,revised]{geophysics}

% Additional packages to geophysics.cls
\usepackage[USenglish]{babel}
\usepackage[utf8]{inputenc}
\usepackage{lmodern}
\usepackage[T1]{fontenc}
\usepackage{amssymb, amsmath, amsfonts}
\usepackage{upquote}
\usepackage[strings]{underscore}
\usepackage{natbib}
\PassOptionsToPackage{hyphens}{url}
\usepackage[pdftex, final, breaklinks=True]{hyperref}
\hypersetup{allcolors=blue, allbordercolors={0 0 .5}, colorlinks=true}

% Figures
\DeclareGraphicsExtensions{.pdf,.png,.jpg}
\renewcommand{\figdir}{./figures}

% Own commands
\newcommand{\mr}[1]{\mathrm{#1}}

\begin{document}

\title{Status of the open-source landscape for 3D CSEM modeling}

\renewcommand{\thefootnote}{\fnsymbol{footnote}}

\ms{GEO-2018-0069.R2}

\address{
\footnotemark[1]TU Delft,
Building 23,
Stevinweg 1 / PO-box 5048,
2628 CN Delft,
E-mail: \href{mailto:D.Werthmuller@tudelft.nl}{D.Werthmuller@tudelft.nl};
}


\author{%
Dieter Werthmüller\footnotemark[1],
Lindsey Heagy,
Raphael Rochlitz, and
Octavio Castillo-Reyes
}

\footer{}
\lefthead{Werthmüller et al.}
\righthead{Digital filter designing tool}

\maketitle

\begin{abstract} % 1-2 sentence(s) each
%
% (1) PRINCIPAL OBJECTIVES AND SCOPE OF THE WORK
%
% (2) METHODOLOGY
%
% (3) RESULTS
%
% (4) CONCLUSIONS
%
To-Do/-Discuss:
\begin{itemize}
  \item Title?
  \item Order of authors? These authors or others?
  \item Other codes to include? Currently considered:\\
    SimPEG, custEM, PETGEM, and emg3d.
  \item Are these references OK?
    \begin{itemize}
      \item \cite{CAG.15.Cockett} (SimPEG)
      \item \cite{GEO.18.Rochlitz} (custEM)
      \item \cite{CAG.18.CastilloReyes} (PETGEM)
      \item \cite{JOSS.19.Werthmuller} (emg3d).
    \end{itemize}
  \item Exact structure of the paper?
  \item Exact models to calculate/show/compare?
\end{itemize}
\end{abstract}

\section{Introduction}

\dots

\section{Codes}

Each code should outline:
\begin{itemize}
  \item Equation system it solves;
  \item the used discretization possibilities;
  \item domains (frequency, time);
  \item details (anisotropy; el. perm. and mag. perm.);
  \item other things (inversion; other methods);
  \item speed and memory estimation;
  \item plans for next features.
\end{itemize}

\subsection{emg3d}

\subsection{SimPEG}

\subsection{custEM}

\subsection{PETGEM}

\section{Numerical Results}

\subsection{Layered model}

A simple layered model, comparing with \texttt{empymod}
\citep{GEO.17.Werthmuller}.

\subsection{3D model}

A complicated, big, 3D model with bathymetry (topography).

\section{Conclusions}

The landscape in 3D CSEM modelling greatly changed in the last five years or
so. While before there were only closed-source codes owned by companies or
consortia (e.g., CEMI) there was recently a wave of openly released codes.
\dots

\section{Acknowledgment}
\dots


% REFERENCES
\bibliographystyle{references}
\bibliography{references}

% Include bbl instead above command before submission.
% \begin{thebibliography}{}
% \itemsep0pt
% \bibitem ...
% \end{thebibliography}

\end{document}

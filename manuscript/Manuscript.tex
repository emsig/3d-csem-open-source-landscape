\documentclass[
    paper,
%     manuscript,
%     twocolumn,
%     twoside,
%     revised,
  ]{geophysics}

% Remove once done
\usepackage[textsize=footnotesize, textwidth=2.8cm]{todonotes}

% Additional packages to geophysics.cls
\usepackage[UKenglish]{babel}
\usepackage[utf8]{inputenc}
\usepackage{lmodern}
\usepackage[T1]{fontenc}

\usepackage{amssymb, amsmath, amsfonts}
\usepackage{upquote}
\usepackage[strings]{underscore}
\usepackage{siunitx}                % SI conform commands (eg \num)
\usepackage{tabularx}
\usepackage{colortbl}
\usepackage{booktabs}
\usepackage{xspace}

\usepackage[pdftex, final]{hyperref}
% \usepackage[pdftex, hidelinks]{hyperref}
\hypersetup{allcolors=blue, allbordercolors={0 0 .5}, colorlinks=true}

% Figures
\DeclareGraphicsExtensions{.pdf,.png,.jpg}
\renewcommand{\figdir}{./figures}
\ifthenelse{\boolean{@twoc}}{%
  \newcommand{\cwidth}{250pt}}{%
  \newcommand{\cwidth}{240pt}}

\ifthenelse{\boolean{@twoc}}{%
  \newcommand{\acwidth}{250pt}%
  \newcommand{\tcwidth}{250pt}}{%
  \newcommand{\acwidth}{240pt}%
  \newcommand{\tcwidth}{312pt}}

% Own commands
\newcommand{\mr}[1]{\mathrm{#1}}
\newcommand{\emg}[2]{\texttt{emg#1#2}\xspace}
\newcommand{\empymod}{\texttt{empymod}\xspace}
\newcommand{\simpeg}{\texttt{SimPEG}\xspace}
\newcommand{\custem}{\texttt{custEM}\xspace}
\newcommand{\petgem}{\texttt{PETGEM}\xspace}

% Remove once done
\newcommand{\itodo}[1]{\todo[color=blue!40!white, inline]{\sffamily #1}}
\newcommand{\mtodo}[1]{\todo[inline]{\sffamily #1}}
\newcommand{\etodo}[1]{\todo[color=blue!40!white]{#1}}
\newcommand{\dul}[1]{\underline{\underline{#1}}}
\newcommand{\ul}[1]{\underline{#1}}
\newcommand{\rmk}[1]{{\color{red}{#1}}}

\begin{document}

\title{Open-source landscape for 3D CSEM modelling}

\renewcommand{\thefootnote}{\fnsymbol{footnote}}

\ms{}  % ARTICLE-ID

\address{
\footnotemark[1]TU Delft,
Building 23,
Stevinweg 1 / PO-box 5048,
2628 CN Delft;\\
\footnotemark[2]TODO\\
E-mail: \href{mailto:Dieter@Werthmuller.org}{Dieter@Werthmuller.org};\\
\textbf{Keywords}: CSEM, Open-Source, 3D modelling, FD, FE.
}


\author{%
Dieter Werthmüller\footnotemark[1], %            orcid: 0000-0002-8575-2484
Raphael Rochlitz\footnotemark[2], %              orcid: 0000-0002-5132-916X
Octavio Castillo-Reyes\footnotemark[2], and %    orcid: 0000-0003-4271-5015
Lindsey Heagy\footnotemark[2] %                  orcid: 0000-0002-1551-5926
}

\footer{}
\lefthead{Werthmüller et al.}
\righthead{Open-source 3D CSEM modelling}

\maketitle

%%fakesection ===    ABSTRACT    ===
\begin{abstract}
%
Write at the end; one or two sentence(s) each:
%
\begin{enumerate}
  \item Principal OBJECTIVES AND SCOPE OF THE WORK
  \item Methodology
  \item Results
  \item Conclusions
\end{enumerate}
%
\end{abstract}

\itodo{
We will have to decide on a Journal. Current suggestions:\\
- GMD: Geoscientific Model Development\\
- SIG: Surveys in Geophysics\\
- SE: Solid Earth\\
- GJI: Geophysical Journal International\\
- GP: Geophysical Prospecting\\
- GEO: Geophysics\\
- CAG: Computers and Geosciences\\
I think \emph{Geoscientific Model Development} and \emph{Surveys in Geophysics}
are the favoured currently. I personally would be very interested in GMD. What
do you think? Also interesting would be the GJI, because the MT comparison was
published there, so it would be in the same place.
}

\section{Introduction}

Controlled-source electromagnetic (CSEM) measurements is a frequently applied
method in exploration geophysics for various applications, such as ground
water, geothermal, oil and gas, mining, civil engineering, or geo-hazards.
Modelling these electromagnetic fields is therefore of great interest in order
to understand the measured data. Publications regarding 3D modelling in
electromagnetic methods started to appear as early as the 1970's and 1980's.
These early publications where all integral equation (IE) methods, having an
anomaly embedded within a layered medium, for loop-loop type transient EM
measurements \citep{GJI.74.Raiche, GEO.75.Hohmann, GJI.82.Das, GEO.86.Newman}
and magnetotelluric (MT) measurements \citep{GEO.84.Wannamaker}.

In the 1990's computer became sufficiently powerful that 3D modelling gained
traction, which resulted amongst other in the publication of the book
\emph{Three-Dimensional Electromagnetics} \citep{B.SEG.99.Oristaglio} by the
SEG. Often cited publications from that time are \cite{RSC.94.Mackie}, 3D MT
computation; \cite{RS.94.Druskin}, frequency- and time-domain modelling using a
Yee grid and a global Krylov subspace approximation; and \cite{RS.96.Alumbaugh,
GJI.97.Newman}, low-to-high frequency computation on massively parallel
computers.

The continuous improvement of computing power and the CSEM boom in the early
2000's in the hydrocarbon industry led to a wealth of publications. The amount
of available numerical solutions can be overwhelming and is part of the reason
why there are hundreds of publications about the topic. There are the different
methods to solve Maxwell's equation, such as the IE method
\citep{GJI.74.Raiche, RS.02.Hursan, GEO.06.Zhdanov, GP.10.Tehrani,
CAG.16.Kruglyakov, MGS.17.Kruglyakov} and different variations of the
differential equation (DE) method, for instance finite differences (FD)
\citep{IEEE.66.Yee, RSC.94.Mackie, RS.94.Druskin, GEO.09.Streich,
CAG.13.Sommer}, finite elementes (FE) \citep{GJI.11.Schwarzbach,
GEO.12.daSilva, GJI.13.Puzyrev, GJI.13.Grayver, SEG.16.Zhang}, finite volume
(FV) \citep{EM.90.Madsen, ECP.07.Haber, GEO.14.Jahandari}, and finite
integration technique (FIT) \citep{PIER.01.Clemens, GP.06.Mulder}. And these
are just the most common ones.

There are also many different types of discretisation, where the most common
ones are regular grids (Cartesian, rectilinear), mostly using a Yee grid
\citep{IEEE.66.Yee} or a Lebedev grid \citep{CMMP.64.Lebedev}, but also
unstructured tetrahedral grids \citep{SEG.16.Zhang, CAG.17.Cai}, OcTree meshes
\citep{ECP.07.Haber}, or hexagonal meshes \citep{CAG.14.Cai}.

The biggest variety of all exists probably in the available solvers to solve
the system of linear equations; direct solvers \citep{GEO.09.Streich,
GEO.15.Grayver, GP.14.Chung, GEO.14.Jaysaval, SEG.15.Oh, GJI.18.Wang}, indirect
solvers \citep{GP.06.Mulder, GJI.15.Jaysaval} or a combination of both,
so-called hybrid solvers \citep{GEO.18.Liu}; the solvers often use
preconditioners such as the multigrid method \citep{SIAM.02.Aruliah,
GP.06.Mulder, GJI.16.Jaysaval}.

A very well written overview up to the year 2005 of the different approaches to
3D EM modelling is given by \cite{SG.05.Avdeev}. In the last 15 years the
publications with regards to 3D EM  modelling grew tremendously, driven
probably by improved computer powers. Avdeev finished his review with the
following statement: «\emph{The most important challenge that faces the EM
community today is to convince software developers to put their 3-D EM forward
and inverse solutions into the public domain, at least after some time. This
would have a strong impact on the whole subject and the developers would
benefit from feedback regarding the real needs of the end-users.}»

It is this topic that we want to address, open-source 3D CSEM codes. Some big
exploration companies and service companies have their own codes, as well as
certain research consortia. But how about codes that are open-source?
Traditionally research code was often available upon request by the author.
Codes distributed that way often come with the request to not share the code,
or they often come without a license attached at all (which means that they are
not open-source). Also, at times there can be a significant hurdle to install
a code.

However, the meaning of open-source evolved rapidly in the last decades. Today
open-source generally not only means that a code is available (with a proper
license). It includes much more today, such as that the code is hosted online,
is under version control, has the possibility to file issues and make pull
requests to fix bugs. Good codes also have continuous integration that includes
extensive testing of the code base and online hosted documentation that
includes the API. As such the term has shifted from purely open-source code
to development in the open and increasingly building a global community. There
are a number of projects within the realm of geophysical exploration, besides
the ones we present here, e.g. pyGIMLi \citep{CAG.17.Rucker} and Fatiando
\citep{JOSS.18.Uieda}.

Newer \todo{wrong term, think about it}programming environments  have also
brought much simplifications in the installation process and have most
importantly simplified it for code developers to make their codes easily
available and installable. All our presented codes are in the Python
environment, and each code can be install with a single command (which includes
downloading and installation).


~\\ \hrule


Mention \cite{GJI.13.Miensopust}, which is a similar thing, but for 3D
modelling and inversion for MT.


It is worth mentioning that there are many more open-source 3D electromagnetic
codes developed in other fields than geophysics. Examples include \dots (TODO:
add at least three references). While these codes could potentially be used for
the same goals as presented here we restrict our review to codes purpose-built
for geophysical applications.

An integral part of this publication are the necessary instructions and codes
to reproduce all published results. In fact, the provided codes contain more
than just the results shown here, as including everything here would have made
the article to lengthy.

This will significantly increase the user base and the number of people who are
actually doing 3D EM modelling.

~\\ \hrule

\begin{itemize}
  \item Why need:
    \begin{itemize}
      \item analytical solutions
      \item semi-analytical solutions such as 1D codes (analytical [recursion]
        in k-f domain followed by numerical transformation).
      \item need of comparison of more complex models.
    \end{itemize}
  \item What exists:
    \begin{itemize}
      \item very quick run-down of codes, methods, discretizations.
      \item Thesis Raphael
      \item LitRev Dieter
    \end{itemize}
  \item What changed:
    \begin{itemize}
      \item Python
      \item Demand/Desire for reproducible research
      \item Open-source has changed:
        \begin{itemize}
          \item easy install
          \item documented
          \item bug/issues
          \item communities
          \item testing/benchmarks/coverage/style/quality
        \end{itemize}
    \end{itemize}
  \item What is this paper:
    \begin{itemize}
      \item 4 codes, 2 FD 2 FE
      \item 2 models, simple one after MT paper, complex Marlim one
    \end{itemize}
  \item Outline of paper
  \item Zenodo, provide all codes to reproduce the results.
\end{itemize}


~\\ \hrule


There are many codes for EM - our focus here is on the application to
controlled-source electromagnetic measurements, even though all of the codes
can handle different setups too.

in such fields as the modelling of communication systems (antennas, radar,
satellites), medical imaging, and others.

1D, 2D, we look at 3D.
time domain and frequency domain


Common methods in geophysics are the

\begin{itemize}
  \item integral equation (IE, \cite{GJI.74.Raiche, GEO.06.Zhdanov,
    CAG.16.Kruglyakov}), (scattering equation) (anomalies in layered
    background), computational effort comparably low
  \item differential equation such as finite difference (FD \cite{IEEE.66.Yee,
    GEO.93.Wang}), simple to implement
  \item and finite volume (FV \cite{EM.90.Madsen, GEO.14.Jahandari}),
  \item or and finite element (FE \cite{GEO.04.Commer GEO.09.Streich})
    (approximating the DE on edges and nodals), good for complex geometries,
  \item finite integration technique (FIT, \cite{AEU.77.Weiland},
    \cite{PIER.01.Clemens}, \cite{GP.06.Mulder})
\end{itemize}

OctTree meshes or severe stretching \citep{ECP.07.Haber}


Elmer FEM multiphysical simulation software P. Råback, P.-L. Forsström, M. Lyly
and M. Gröhn, Elmer - finite element package for the solution of partial
differential equations, poster presentation, EGEE User Forum, 2007, Manchester,
UK. http://www.csc.fi/elmer


Many commercial, e.g., COMSOL Multiphysics; EMGS, RSI, KMS; the oil and service
companies.

Consortia (CEMI; Key)

Universities, e.g. Pottsdam (Streich, Grayver)

Computational cost IE < FD < FV < FE

Direct solvers vs indirect solvers, both its advantages and disadvantages;
memory, various sources.

Grids (regular grids, unstructured grids, Octree, tetrahedra, hexahedral,
adaptive, multigrids).

Boundary condition



\section{Codes}

Intro to it; numerical approximations (FE, FD), differences in solvers,
gridding/meshing, boundary conditions, etc.

It is beyond the scope of this paper to go into every detail of the different
modellers, however, a quick blabla...

All the codes presented here
\begin{itemize}
  \item are \emph{just available} (you don't have to write someone and ask if
    you could have the code);
  \item are \emph{easy installable} (through \texttt{pip} or \texttt{conda});
  \item are under \emph{version control} and online either on GitHub or on
    GitLab, which includes possibilities of easy involvement by either creating
    an issue, reporting a bug, or filing a pull request; and
  \item are well documented and come with a lot of examples.
\end{itemize}

\begin{itemize}
  \item Code (license): \emg3d (Apache-2.0), custEM (LPGL-3.0), PETGEM
    (GPL-3.0), and \simpeg (MIT).
  \item Hosting: GitHub (\emg3d, \petgem, \simpeg); GitLab (\custem)
\end{itemize}




Each code should outline:

\begin{itemize}
  \item Equation system it solves;
  \item the used discretization possibilities;
  \item domains (frequency, time);
  \item details (anisotropy; el. perm. and mag. perm.);
  \item other things (inversion; other methods);
  \item speed and memory estimation;
  \item plans for next features.
  \item Inversion capabilities
  \item Time/frequency
  \item transmitter types
\end{itemize}

Basically summarize each code in three points:
\begin{itemize}
  \item Brief intro and main features
  \item Main selling point
  \item Planned features
\end{itemize}

Cite for all: NumPy, SciPy.

The codes under consideration are presented here in alphabetical order.

\subsection{custEM}

\cite{GEO.19.Rochlitz}

FEniCS \citep{CSE.15.Alnaes}


\subsection{emg3d}

The 3D CSEM modeller \emg3d \citep{JOSS.19.Werthmuller} is a multigrid
\citep{CMMP.64.Fedorenko} solver for 3D electromagnetic diffusion following
\cite{GP.06.Mulder}, with tri-axial electrical anisotropy and isotropic
magnetic permeability. The matrix-free solver can be used as main solver or as
preconditioner for one of the Krylov subspace methods implemented in SciPy
\citep{NM.20.Virtanen}, and the governing equations are discretized on a
staggered grid by the finite-integration technique \cite{AEU.77.Weiland}, which
can be viewed as a finite-volume generalization of \cite{IEEE.66.Yee}'s scheme.
The code is written completely in Python using the NumPy
\citep{CSE.11.VanDerWalt} and SciPy stacks, where the most time- and
memory-consuming parts are sped up through jitted functions using Numba
\citep{LLVM.15.Lam}.

The multigrid method is characterized by almost linear scaling both in terms
of CPU usage and RAM, and it is therefore a comparably low-memory consumption
solver (see results). However, the way multigrid is implemented also means some
additional constraints on the chosen grid. You can input any three-dimensional
grid into \emg3d, but the implemented method works with the existing nodes,
meaning there are no new nodes created as coarsening is done by combining
adjacent cells. The more times the grid dimension can be divided by two the
better it is suited. Ideally, the dimension of the coarsest grid should be a
low prime, and grid dimensions are therefore given by their multiplication with
powers of two (${2,3,5}·2^n$).

Currently \emg3d can be used to model the electric and magnetic fields due to
arbitrary finite length electric dipole sources for a single frequency at a
time. Helper functions to compute time-domain responses are in the works
(should be ready by publishing this article). Current development is focused
on adding inversion capabilities; averaging routines to implement strong
topography; and general anisotropy.

The 3D solver \emg3d can be found in the GitHub organization of \empymod
(`empymod.github.io <https://empymod.github.io>`_). The modeller \empymod
\citep{GEO.17.Werthmuller} is a 1D code which can compute electric or
magnetic responses due to a 3D electric or magnetic source in a layered-earth
model with vertical transverse isotropic (VTI) resistivity, VTI electric
permittivity, and VTI magnetic permeability, from very low frequencies (DC) to
very high frequencies (GPR).

In the future it should be possible to use \emg3d as a solver within the
\simpeg framework.


\subsection{PETGEM}

\cite{GJI.19.CastilloReyes}

FEniCS \citep{CSE.15.Alnaes}


\subsection{SimPEG}

\cite{CAG.15.Cockett}

PARDISO \citep{FGCS.04.Schenk}



\section{Comparison}

\subsection{Block Model}

Simple block (one to three blocks) model similar to \cite{GJI.13.Miensopust};
potentially with a bit a funkier source (say, finite length rotated bipole and
finite length receivers). In this case I would only compare inline and
crossline responses.

\begin{itemize}
  \item VTI resistivity
  \item \simpeg => use OcTree for this example? That would be awesome!
  \item Show Ex, Ey, Ez, Hx, Hy, Hz
\end{itemize}

Tables:

%
\tabl[btp]{comp1}{Caption.}{
  \centering
\begin{tabular}{lccrS[table-format=2.4]rl}
  \toprule
  %
  Code & Method & Mesh & \#Cells & {CPU} & RAM & Computer \\
  \midrule
  %
  \custem & FE  & FE & & & & \\
  \emg3d  & FIT & FD & & & & \\
  \petgem & FE  & FE & & & & \\
  \simpeg & FD  & OcTree & & & & \\
  %
  \bottomrule
\end{tabular}}%
%

%
\plot*{block-model}{width=.6\textwidth}{
   Block model, consisting of a layered background, in which three blocks are
   embedded.
}
%

%
\plot*{layered-result}{width=.6\textwidth}{
   Comparison of the layered model.
}
%

%
\plot*{block-result}{width=.6\textwidth}{
   Comparison of the block model.
}
%


\subsection{Marlim R3D}

The Marlim R3D model is a \dots \cite{BJG.17.Carvalho} (model)
\cite{GEO.19.Correa} (CSEM data)

Comparison to an code from the industry, using \emph{SBLwiz} from EMGS.



\section{Discussion}

The landscape in 3D CSEM modelling greatly changed in the last five years or
so. While before there were only closed-source codes owned by companies or
consortia (e.g., CEMI), or codes that you had to obtain from the authors, often
without much documentation, there was recently a wave of openly released codes.

\dots

We see this only as the start. Much more comparisons and examples are required.
3D CSEM modelling is a difficult task, which requires many considerations: It
starts with the selection of the right code for the problem; then the meshing
is particularly difficult, choosing cells small enough to appropriately
represent the model yet to be as coarse as possible still achieving the desired
precision; the required model extent, particularly for shallow and marine cases
where the airwave has to be considered. The tricky part is that even if the
result is not correct it may look completely valid. The only options to verify
results are (a) by comparing different discretizations, and (b) by comparing
different codes.

Here we only consider frequency-domain results from frequency-domain
computations. Similar comparison for time-domain results from time-domain
computations, and frequency-domain results from time-domain computations and
vice versa using Fourier transforms would also be good.

The readily availability of many codes means there will be many more people
able to model 3D results, and not just a few specialists in the field. Whilst
this is amazing and will push the field much further it also means that the
same mistakes might happen over and over again, and comparisons and examples
like these can help to avoid this.

\subsection{Future (?)}

Comment KK: \emph{One thing that the community could really use is more test
models for various scenarios, and having them be easily accessible.}


\subsection{MARE3DEM}

This is, I thought, the right spot where Kerry Key could write a bit about
MARE3DEM, if (and I think only if) there are clear plans and a time-scale to
open-source it. Embedded in a wider summary of available codes (some research
required).

\section{Conclusions}

The landscape in 3D CSEM modelling greatly changed in the last five years or
so. While before there were only closed-source codes owned by companies or
consortia (e.g., CEMI) there was recently a wave of openly released codes.

Results from numerical modellers for simple models, such as a halfspace below
air or a sphere in a fullspace, can be verified by comparing them to analytical
solutions. However, there is no such possibility for complicated, realistic
models. The only way to gain confidence in numerical results is by comparing
different codes to each other. We hope that our efforts help to the confidence
in the available open-source codes\dots

Also, the runtime and memory consumption comparison together with the meshing
differences should help to decide which modeller is used best for which tasks,
as there is not one method which is best for all problems.

It is important to note that there exist many more 3D electromagnetic codes
from various fields, such as civil engineering or medicine, in Python and in
many other languages (list a few references here). Here we focused on codes in
Python developed for geophysical exploration.

\dots

\section{Acknowledgment}

\itodo{Write here your fundings; and potentially also minor contributors etc.}

We would like to thank Paulo Menezes for the help and explanations with regards
to the Marlim R3D model and corresponding CSEM data, and for making their
actual computation model available under an open-source license.

The work of D.W. was conducted within the Gitaro.JIM project funded through
MarTERA, a EU Horizon 2020 research and innovation programme (No 728053);
\href{https://www.martera.eu}{martera.eu}.


% REFERENCES

\bibliographystyle{Refs}          % Modified SEG bibliography style.
\bibliography{Refs}

\end{document}
